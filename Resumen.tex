\chapter{Resumen}
%\addcontentsline{toc}{chapter}{Resumen}
%\titleformat{\section}{\normalfont\Large\bfseries}{}{0pt}{}

\markboth{Resumen}{}

{\setlength{\leftskip}{10mm}
\setlength{\parindent}{-10mm}

\autor.

Candidato para obtener el grado de \grado\orientacion.

\uanl.

\fime.

Título del estudio: \textsc{\titulo}.

Número de páginas: 214.
%\noindent Número de páginas: 214.

%%% Comienza a llenar aquí
\paragraph{Objetivos y método de estudio:}
Encontrar entre los métodos existentes de interpolación espacial, el método que mejor pronostique niveles de contaminación del aire en el Área Metropolitana de Monterrey. Los métodos seleccionados son: Teselaciones de Voronoi, Distancia Inversa Ponderada, Funciones de Base Radial y Kriging, los cuales se utilizan para pronosticar las variables Ozono, Monóxido de Carbono, Óxido Nítrico, Bióxido de Nitrógeno, Óxidos de Nitrógeno, Partículas menores a 10 micras, Partículas menores a 2.5 micras y Bióxido de Azufre. Dichos datos de las variables son proporcionados por el Sistema Integral de Monitoreo Ambiental de la Secretaría de Desarrollo Sustentable del Gobierno de Nuevo León.

\paragraph{Contribuciones y conlusiones:}
Generar mapas que describan la calidad del aire sobre toda la área de estudio para cada contaminante, ya que cada variable sigue una distribución de los datos diferente a la del resto y así evaluar el Índice de {\sc AIRE} y {\sc SALUD} propuesto en la Norma Oficial Mexicana NOM-020-SSA1-2014, el cual indica los riesgos asociados a los niveles de contaminación de cada contaminante del aire.

\bigskip\noindent\begin{tabular}{lc}
\vspace*{-2mm}\hspace*{-2mm}Firma del asesor: & \\
\cline{2-2} & \hspace*{1em}\asesor\hspace*{1em}
\end{tabular}


